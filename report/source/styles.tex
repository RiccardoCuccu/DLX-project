\lstdefinelanguage{VHDL}{morekeywords={library,use,all,entity,generic, is,port,in,out,end,architecture,of,begin,and,if,then,else,elsif,process},morecomment=[l]--}

%\lstdefinestyle{vhdl}{language = VHDL, basicstyle = \ttfamily, keywordstyle = \color{keyword}\bfseries, commentstyle = \color{comment}}

\lstdefinestyle{MyVHDL} {
    language=VHDL,
    basicstyle=\scriptsize\ttfamily,
    frame=lines,
    tabsize=8,
    breaklines,
    captionpos=b,
    backgroundcolor=\blendcolors*{!25!white}\color{lightgray},
    keywordstyle=\color{blue},
    stringstyle=\color{orange},
    commentstyle=\color{gray},
    morecomment=[l][\color{gray}]{--}
}

\lstdefinestyle{MyTcl} {
    language=tcl,
    basicstyle=\scriptsize\ttfamily,
    frame=lines,
    tabsize=8,
    breaklines,
    captionpos=b,
    backgroundcolor=\blendcolors*{!25!white}\color{lightgray},
    keywordstyle=\color{blue},
    stringstyle=\color{orange},
    commentstyle=\color{gray},
    morecomment=[l][\color{gray}]{\#}
}

\lstdefinestyle{MyBash} {
    language=bash,
    basicstyle=\scriptsize\ttfamily,
    frame=lines,
    tabsize=8,
    breaklines,
    captionpos=b,
    backgroundcolor=\blendcolors*{!25!white}\color{lightgray},
    keywordstyle=\color{blue},
    stringstyle=\color{orange},
    commentstyle=\color{gray},
    morecomment=[l][\color{gray}]{\#}
    showstringspaces=false,
}

\lstdefinestyle{MyShell} {
    language=bash,
    basicstyle=\scriptsize\ttfamily\color{white},
    frame=lines,
    tabsize=8,
    breaklines,
    captionpos=b,
    backgroundcolor=\color{black},
    keywordstyle=\color{cyan},
    stringstyle=\color{orange},
    commentstyle=\color{green},
    showstringspaces=false,
}